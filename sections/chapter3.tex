%%% BEGIN CHAPTER 3 Design of fast folding potassium channel mutant %%%

%% Paper title page %%
\chapter{Biochemical preparation of KcsA monomers and Design of fast folding KcsA mutant}
\section{Introduction}
Bacterial potassium channel, KcsA, is an extremely stable tetramer. Conventional denaturing reagents such as urea and guanadine do not unfold KcsA. Sodium dodecyl sulfate (SDS) detergent is known to be a very harsh denaturing reagent; however, even SDS alone does not unfold KcsA. However, with 0.5\% SDS and incubation of the protein at 90 $^{\circ}$C for more than 10 minutes, the KcsA tetramer can be disrupted. In the first part of this chapter, a protocol for preparing KcsA monomer is presented. We found that the WT KcsA monomer is highly dynamic and exists in a structurally diverse ensemble of states. To stabilize the monomeric state, we introduce a disulfide bridge at the end of the two TM helices. This disulfide-bonded (CC) KcsA mutant monomer displays a much more native-like structure and is much more efficient at refolding into tetramers than the WT KcsA monomer. 

\section{Methods}
\subsection{KcsA expression and purification}
For protein expression, XL10-GOLD competent cells were transformed with pQE70 KcsA WT with C-terminal 6xHis-tag and grown overnight at 37 $^{\circ}$C with ampicillin (100 $\mu$g/mL). This overnight culture was used to inoculate LB media and ampicillin (100 $\mu$g/mL) at a final concentration of 1\% v/v. Cultures were grown until OD$_{600}$ reached 0.6 and induced with 1 mM Isopropyl $\beta$-D-1-thiogalactopyranoside (IPTG). Then, cells were harvested 3 –- 4 hours later by centrifuging at 9,000 RCF for 15 minutes at 4 $^{\circ}$C. For NMR samples, cultures were grown until OD$_{600}$ reached 0.6 and were spun down at 4,000 RCF for 10 minutes at 4 $^{\circ}$C. The biomass was doubled by combining cell pellets from 2 L of LB media into 1 L of M9 media. The pellet was resuspended with 20 mL of M9 media and returned to 1 L of M9 media. The cells were allowed to acclimate for an hour at 37 $^{\circ}$C and then were induced with 1 mM IPTG. Pellets were harvested 3 –- 5 hours later by spinning down at 9,000 RCF for 15 minutes at 4 $^{\circ}$C. After the pellet was spun down, the pellet was frozen in -70 $^{\circ}$C until purification. In a given liter, M9 media consisted of 3 g KH$_{2}$PO$_{4}$, 12.8 g Na$_{2}$HPO$_{4}$, 0.5 g NaCl, 0.1 g MgSO$_{4}$, 1 g $^{15}$NH$_{4}$CL, 3 g glucose, 0.01 g CaCl$_{2}$, 0.01 g thiamine, 0.01 g FeSO$_{4}$, and 0.05 volume of multivitamin tablet (CVS).

For purification, the frozen pellets were resuspended in buffer A (50 mM HEPES, 200 mM KCl, pH 7.0) with 1 mM phenylmethylsulfonyl fluoride (PMSF), 3 mg of DNAse A (Goldbio) per liter of culture, and 0.4 mM MgSO4. The cells were lysed through 3x passage through french press, and the lysed cells were spun down at 158,000 RCF for 30 minutes at 4 $^{\circ}$C. The pellet was resuspended in buffer A with 10 mM DDM and 1 mM PMSF. The mixture was rotated at room temperature for 1 hour to extract and solubilize KcsA in DDM. Then, the mixture was spun down at 185,000 RCF for 1 hour at 4 $^{\circ}$C. Supernatant was incubated with Talon Metal Affinity Co2+ resins and rotated at room temperature for 1 hour. The resins were then collected by gravity and the flow through was discarded. The columns were washed with 15 bed volumes of Buffer A + 1 mM DDM + 10 mM imidazole and the proteins were eluted with Buffer A + 1 mM DDM + 500 mM imidazole.

For full-length KcsA constructs, protein in elution solution was further purified by size-exclusion chromatography on Bio-rad SEC 650 enrich column that was pre-equilibrated with Buffer containing 50 mM NaPi, 100 mM NaCl, 1.5 mM DDM, pH 6.5. For KcsA $\Delta$125 constructs, full length KcsA constructs were trypsinized with $\alpha$-chymotrypsin (Sigma Aldrich) at 4 $^{\circ}$C overnight with 1:50 (chymotrypsin:KcsA) ratio. Then, the digested KcsA was further purified through size-exclusion chromatography (SEC) with Superdex 200 Increase column. 

\subsection{Design fast folding KcsA mutant}
Based on previous simulations of WT Kv1.2 and KcsA monomers as well as NMR results of WT KcsA monomers, KcsA monomer seemed to be highly dynamic and structurally heterogeneous. In order to limit the dynamical nature of KcsA WT monomer, we introduce a disulfide bridge at the end of the two TM helices to limit its dynamical nature. The residues chosen for mutations are A29 and A109. These sites were chosen based on looking at the crystal structure of KcsA

\subsection{KcsA monomer preparation}
KcsA constructs after SEC were pooled and precipitated by adding 10\% tricholoroacetic acid (TCA) to the samples. The precipitated samples were frozen at -80 $^{\circ}$C for 30 minutes. Then, the frozen samples were spun down at 16,200 RCF for 30 minutes at 4 $^{\circ}$C. Supernatant was removed, and washed with chilled (-20 $^{\circ}$C) acetone. Precipitates were resuspended, vortexed for 30 seconds, and incubated at -20 $^{\circ}$C for 30 minutes until spin down at 16,200 RCF for 15 minutes at 4 $^{\circ}$C. This acetone wash step was repeated 3 times. The final precipitate was dried under air until all acetone evaporated, and the precipitate was stored in -80 $^{\circ}$C until usage.

\subsection{Nuclear magnetic resonance (NMR) measurements of KcsA}
NMR samples were put in q=0.3 DMPC:DHPC bicelles using following procedure. Bicelle mixture was first prepared by adding 1:3 molar ratio of DMPC to DHPC (both solubilized in chloroform) at final lipid concentration of 10\% w/v in the NMR sample. Chloroform was evaporated under stream of nitrogen and further removed under vacuum for 3 – 4 hours. This procedure resulted in a thin lipid film, which was resolubilized in 2,2,2-trifluoroethanol (TFE). WT KcsA $\Delta$125 precipitate is solubilized in TFE and mixed into the lipid TFE mixture. TFE was evaporated under a stream of nitrogen until a thin film formed around the glass tube and was further evaporated by placing the tube under vacuum overnight. NMR spectra were acquired on Bruker AVANCE IIIHD 600 MHz NMR spectrometer equipped with a room temperature TXI probe. [$^{1}$H, $^{15}$N]-TROSY HSQC experiments were run at T = 323 K. To obtain rotational correlation times, $\tau_{c}$, [15N, 1H]-TRACT experiments were performed.74-77 

\subsection{KcsA CC simulations and MSM analysis}
The starting structure for KcsA monomer simulations was taken from the tetrameric KcsA X-ray crystal structure (PDB ID: 1R3J). Mutations were made at A29C and A109C for the disulfide mutant simulations, and disulfide bond was created using CHARMM-GUI’s PDBReader module. All systems were prepared by using CHARMM-GUI’s Membrane Builder module (www.charmm-gui.org)64-68. Each system contained 70 1,2-dimyristoyl-sn-glycero-3-phosphatidylcholine (DMPC) molecules per leaflet totaling up to 140 DMPC molecules to match the NMR sample conditions. All systems were hydrated by creating a water box 17.5 Å above and below the protein’s maximum and minimum Z-positions. In addition, the system was neutralized with 150 mM KCl. The system comprised a total of approximately 45,000 atoms. 5 independent simulations of KcsA WT monomer and 5 simulations of KcsA CC mutant monomer were carried out at T = 353 K to enhance sampling. All simulations were run with the parameters described in Methods section with AMBER16 GPU, and each simulation was run up to 6 $\mu$s. In total 30 $\mu$s of KcsA WT simulations and 30 $\mu$s of KcsA CC simulations were accumulated.

For MSM analysis, all KcsA simulations were first projected onto the TIC space created by Kv1.2 simulations. KcsA monomer has 103 residues in total whereas Kv1.2 pore domain has 99 residues. In order to project KcsA monomer simulations, C$\alpha$ atoms were aligned to Kv1.2 structure. The final corresponding residues in KcsA were from residues 24 to 122, which were used for TICA projection. After projecting KcsA simulations onto the same TIC space as Kv1.2, all simulations were combined in order to create the same number of microstates using K-Center clustering algorithm. After the same microstates were created, Markov state model constructions were done as described in SI Methods. However, MSM was constructed separately for Kv1.2, KcsA WT and KcsA CC using the same set of microstates.

\section{Results and Discussion}
\subsection{Preparation of potassium channel monomers}
\subsection{}
\section{Conclusion}

%% REVERT FIGURE NUMBERING %%
\renewcommand\thefigure{\thechapter.\arabic{figure}} 

%%% END CHAPTER 2 MICROARRAY %%%

